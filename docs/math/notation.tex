\section{Notation}

\subsection{Reference Frames}

\begin{itemize}
  \item $\mathcal{F}_b$: body-fixed frame attached to the robot.
  \item $\mathcal{F}_n$: navigation frame, defined as ENU (East--North--Up).
\end{itemize}

$\bR_{nb} \in SO(3)$ denotes the rotation matrix from body frame to navigation frame.

\subsection{State and Control}

The control-oriented system state is defined as
\begin{equation}
\bx =
\begin{bmatrix}
v_x & v_y & v_z & \omega_z & a_x & a_y & a_z & \psi
\end{bmatrix}^\top,
\end{equation}
where
\begin{itemize}
  \item $(v_x, v_y, v_z)$: linear velocity expressed in the navigation frame $\mathcal{F}_n$,
  \item $\omega_z$: yaw angular rate,
  \item $(a_x, a_y, a_z)$: linear acceleration,
  \item $\psi$: yaw angle.
\end{itemize}

The control input is denoted by
\begin{equation}
\bu \in \mathbb{R}^m,
\end{equation}
representing normalized thruster commands.

\subsection{Temporal Parameters}

\begin{itemize}
  \item Control frequency: $f_c = 50\,\mathrm{Hz}$.
  \item History window length: $L$.
  \item Prediction horizon: $H$.
\end{itemize}

\subsection{Uncertainty}

For any predicted variable $y$, the model outputs both mean $\hat{y}$ and standard deviation $\hat{\sigma}_y$.
